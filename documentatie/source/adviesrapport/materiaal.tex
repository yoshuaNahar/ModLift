\chapter{Materiaal}
\section{Hout}
Hout is een materiaal dat al heel lang wordt gebruikt met bouwen en dat is met een goede reden: het is sterk, redelijk flexiebel maar toch stevig en goedkoop. Het vraagt wel wat onderhoud om het voor een langere periode te kunnen gebruiken (lakken, schuren et cetera) maar voor een prototype is dit irrelevant. Een klein nadeel is dat met dikkere platen van bijvoorbeeld 6mm de plaat kan verbranden in de snijder. Dit vormt op zich geen gevaar als de machine goed ingesteld staat, maar voor zowel de uitstraling als de functionaliteit kan dit hinderend zijn.
\section{Plexiglas}
Plexiglas is een materiaal gemaakt uit een vorm van plastic die zeer geschikt is voor het bouwen van producten. Het is niet alleen onderhoudsvrij maar ook verkrijgbaar in veel verschillende kleuren en maten. Daarnaast is de grote reden dat wij dit materiaal willen gaan gebruiken dat het ook transparant uitgevoerd wordt. Hierdoor kan de bedrading en elektronische schakelingen makkelijk worden bekeken, begrepen en gerepliceerd worden voor eerstejaars. Het is dus niet belangrijk voor eerstejaars om dit materiaal te gaan gebruiken omdat deze niet de schakelingen hoeven te laten zien.

\section{Voor- en nadelen}
\begin{center}
\begin{tabular}{l|c|c}
& Hout & Plexiglas \\
\hline
Stevigheid & \cmark & \cmark \\
\hline
Doorzichtig & \xmark & \cmark \\
\hline
Prijs & \euro{3} & \euro{8}
\end{tabular}
\end{center}

\section{Advies}
Hierbij adviseren wij om de gehele lift van hout te maken. Plexiglas is niet meer vereist omdat onze elektronica zich voornamelijk aan de buiten kan zal bevinden en deze dus toch wel bekeken kan worden. Daarnaast is hout een stuk goedkoper.