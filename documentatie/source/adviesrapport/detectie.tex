\chapter{Lift detectie}

\section{IR: infrarood}
De infrarood sensor kan zien of het licht dat de sensor uitstuurt terugkomt op de sensor. Door bijvoorbeeld een reflector te plaatsen op de lift kan een verdieping zien of het licht terugkomt via de reflector en dus of de lift aanwezig is op dat moment of niet. Dit systeem is makkelijk te implementeren en in te programmeren.

\section{Limit switch}
Een limit switch is een knop die wordt ingedrukt als de lift er langs gaat. De lift drukt de switch in op het moment dat de lift de hoogte van de switch bereikt. Deze methode is zeer betrouwbaar omdat als een switch eenmaal is geïnstalleerd en vast zit, het weinig defecten kan vertonen. Het installeren van deze switch is wat moeilijker omdat de liftkooi goed vast op geleiders moet zitten zodat de switch daadwerkelijk ingedrukt wordt als de lift langskomt en dat de lift niet vast komt te zitten omdat de switch de lift uit de geleiders drukt.

\section{Magnetische reedcontacten}
Magnetische reedcontacten werken door een stuk metaal op de lift te plaatsen die kan worden gedetecteerd door een sensor op de verdieping. Het moment dat de lift langs komt zal de magneetsensor het stukje metaal zien en kan de lift gedetecteerd worden. Een magnetisch reedcontact kan een stukje metaal detecteren zelfs als de lift niet volledig stabiel door de schacht gaat.

\section{Voor- en nadelen}
\begin{center}
\begin{tabular}{l|c|c|c}
& IR & Limit switch & Magnetisch \\
\hline
Betrouwbaar & \cmark & \cmark & \cmark \\
\hline
Makkelijk implementeerbaar & \cmark & \xmark & \cmark \\
\hline
Precisie detectie & 5 mm & 1 mm & 20 mm \\
\hline
Prijs & \euro{7,50} & \euro{10} & \euro{7,50} \\

\end{tabular}
\end{center}

\section{Advies}
Voor de lift detectie adviseren wij infrarood aan, omdat infrarood zowel betrouwbaar is als makkelijk te implementeren.